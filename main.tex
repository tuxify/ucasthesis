%%================================================
%% File: main.tex
%% Encoding: UTF-8
%% Author: Yuan Xiaoshuai - yxshuai@gmail.com
%% Created: 2017-05-25 10:18
%% Last modified: 2019-11-14 22:10
%%================================================
\documentclass[degree=doctor]{ucasthesis}
% \documentclass[degree=doctor]{ucasthesis}
% 选项:
%   degree=[master|doctor],                   % 必选
%   secret,                                   % 可选
%   mkshuji,                                  % 可选
% 注意:mkshuji 选项需要系统中安装 Noto Sans CJK SC 字体

% 所有其它可能用到的包都统一放到这里了,可以根据自己的实际添加或者删除。
\usepackage{ucasthesis}

% 定义所有的图片文件在 figures 子目录下
\graphicspath{{figures/}}

% 可以在这里修改配置文件中的定义。导言区可以使用中文。
% \def\myname{赵钱孙}

\begin{document}

%%% 封面部分
\frontmatter
%%================================================
%% File: cover.tex
%% Encoding: UTF-8
%% Author: Yuan Xiaoshuai - yxshuai@gmail.com
%% Created: 2017-05-25 10:18
%% Last modified: 2019-11-14 16:05
%%================================================
\ucassetup{
  %******************************
  % 注意:
  %   1. 配置里面不要出现空行
  %   2. 不需要的配置信息可以删除
  %******************************
  %
  %=========
  % 中文信息
  %=========
  ctitle={中国科学院大学学位论文 \LaTeX\ 模板\\使用示例文档 v\version},
  ctitlemark={中国科学院大学学位论文 LaTeX 模板使用示例文档 v\version},
  cdegree={工学博士},
  cdepartment={中国科学院大连化学物理研究所},
  cmajor={化学工程},
  cauthor={赵钱孙},
  csupervisor={吴郑王\hspace{2\ccwd}教\hspace{\ccwd}授},
  csupervisorplace={中国科学院大连化学物理研究所},
  ccosupervisor={李\hspace{\ccwd}周\hspace{2\ccwd}研究员}, % 辅助老师
  % 日期自动使用当前时间,若需指定按如下方式修改:(夏季毕业填写6月,冬季填写12月)
  cdate={2018 年 6 月},
  % 致谢日期
  ackdate={2018 年 6 月于某地}, 
  %
  %=========
  % 英文信息
  %=========
  etitle={An Introduction to \LaTeX{} Thesis\\ Template of University of Chinese Academy of Sciences\\ v\version},
  etitlemark={An Introduction to LaTeX Thesis Template of University of Chinese Academy of Sciences v\version},
  edegree={Doctor of Engineering},
  emajor={Chemical Engineering},
  eauthor={Qiansun Zhao},
  esupervisor={Professor Zhengwang Wu},
  ecosupervisor={Professor Zhou Li},
  edepartment={Dalian Institute of Chemical Physics, Chinese Academy of Sciences},
  % 日期自动生成,若需指定按如下方式修改:
  edate={June, 2018}
}
% \input{data/ack}
%%================================================
%% Filename: abstract.tex
%% Encoding: UTF-8
%% Author: Yuan Xiaoshuai - yxshuai@gmail.com
%% Created: 2012-04-24 00:21
%% Last modified: 2019-11-10 10:26
%%================================================
\begin{cabstract}

% 作者所在研究所要求中文摘要内容不能超过一页,稍微多于一页时可以调整行距实现
% \xiaosi[1.72] % fit the abstract in one page

\ucasthesis{}(中国科学院大学学位论文模板)提供中国科学院大学研究生学位论文的模板,代码来源于清华大学学位论文模板(\textsc{ThuThesis}),并根据中国科学院大学学位论文写作规范格式的要求进行了修改。该模板为作者撰写博士论文所用,且通过了格式审查,但由于国科大各研究院所对格式的要求可能存在差异,可能需要根据具体研究院所的格式要求进行一定的调整。

本文的创新点主要有:

  \begin{itemize}
    \item 用例子来解释模板的使用方法;
    \item 用废话来填充无关紧要的部分;
    \item 一边学习摸索一边编写新代码。
  \end{itemize}

\textsf{模板作者注}:关键词分隔符用半角逗号,模板会自动处理替换为《规范》中
规定的分隔符,英文关键词同理。

\end{cabstract}

\ckeywords{TeX/LaTeX, XeLaTeX与中文处理, 科技排版, 国科大, 学位论文
模板, 关于摘要}

\begin{eabstract} 

An abstract of a dissertation is a summary and extraction of research work and
contributions. Included in an abstract should be description of research topic
and research objective, brief introduction to methodology and research
process, and summarization of conclusion and contributions of the research. An
abstract should be characterized by independence and clarity and carry
identical information with the dissertation. It should be such that the
general idea and major contributions of the dissertation are conveyed without
reading the dissertation. 

\end{eabstract}

\ekeywords{TeX/LaTeX, XeLaTeX Chinese, Scientific typesetting system,
Academic thesis template, UCAS, About keywords}



\makecover

%% 目录
\tableofcontents
\listoffigures* % 星号表示不在目录中显示图表目录
\listoftables*

%% 符号对照表
% 规范未限定放置顺序,但研究生部要求放文后
% 鉴于各研究院所要求不同,可以试着放前面,页眉页脚会自动调整
% \input{data/denotation}

%%% 正文部分
\mainmatter
%%================================================
%% Filename: chap01.tex
%% Encoding: UTF-8
%% Author: Yuan Xiaoshuai - yxshuai@gmail.com
%% Created: 2019-11-10 10:30
%% Last modified: 2019-11-10 10:34
%%================================================
\chapter{模板说明 Introduction}
\label{cha:introduction}




%%================================================
%% Filename: chap02.tex
%% Encoding: UTF-8
%% Author: Yuan Xiaoshuai - yxshuai@gmail.com
%% Created: 2019-11-10 10:32
%% Last modified: 2019-11-21 15:54
%%================================================
\chapter{模板使用示例 Some Examples}
\label{cha:examples}

\section{插图示例}
\label{sec:figure}

\subsection{插入单幅图形}

插图通常需要占据大块空白,所以在文字处理软件中用户经常需要调整插图的位置。
\LaTeX~有一个 \texttt{figure} 环境可以自动完成这样的任务,这种自动调整位置的环
境称作浮动环境(float),之后还会介绍表格浮动环境。

单张图片插入形式如图~\ref{fig:golfer}~所示。
\begin{figure}[htbp]
\centering
\includegraphics[width = 0.3\textwidth]{golfer}
\bicaption{打高尔夫球的人}
          {Golfer}
\label{fig:golfer}
\end{figure}

\subsection{插入多幅图形}

\subsubsection*{并排摆放,共享标题}

当我们需要两幅图片并排摆放,并共享标题时,可以在 \texttt{figure} 环境中使用两个
插图命令,如图~\ref{fig:fanqingfuming}~所示。

\begin{figure}[htbp]
\centering
\includegraphics[width=0.3\textwidth]{qingming}
\hspace{36pt}
\includegraphics[width=0.3\textwidth]{fanfu}
\bicaption{反清复明}{Fan Qing Fu Ming}
\label{fig:fanqingfuming}
\end{figure}

\subsubsection*{并排摆放,各有标题}

如果想要两幅并排的图片各有自己的标题,可以在 \texttt{figure} 环境中使用两个
 \texttt{minipage} 环境,每个环境里插入一个图,如图~\ref{fig:qingming}~和
~\ref{fig:fanfu}~所示。

\begin{figure}[htbp]
\centering
\begin{minipage}[t]{0.3\textwidth}
    \centering
    \includegraphics[width=\textwidth]{qingming}
    \bicaption{清明}{Qing Ming}
    \label{fig:qingming}
\end{minipage}
\hspace{36pt}
\begin{minipage}[t]{0.3\textwidth}
    \centering
    \includegraphics[width=\textwidth]{fanfu}
    \bicaption{反复}{Fan Fu}
    \label{fig:fanfu}
\end{minipage}
\end{figure}

\subsubsection*{并排摆放,共享标题,各有子标题}

如果想要两幅并排的图片共享一个标题,并各有自己的子标题,可以使用 \textbf{subcaption} 宏包提供的 \texttt{subcaptionbox} 命令,每个子图可以有各自的引用,如图~\ref{fig:subfig_a}~和~\ref{fig:subfig_b}~所示。

\begin{figure}[htbp]
\centering
\bisubcaptionbox{清明\label{fig:subfig_a}}{Qing Ming}[0.3\textwidth]{
    \includegraphics[width=0.3\textwidth]{qingming}
}
\hspace{36pt}
\bisubcaptionbox{反复\label{fig:subfig_b}}{Fan Fu}[0.3\textwidth]{
    \includegraphics[width=0.3\textwidth]{fanfu}
}
\bicaption{反清复明}{Fan Qing Fu Ming}
\end{figure}

\subsection{乱序假文}

大落亲算古。判独性以深刚里索油银场怀苏军风正!救务图规作微智备看封底雪就居皇用。言征碃研却苏里翻连正吧代欢花。怪街旧持用节八莫往终组洛礼?确女要老值套布伯进育掉青据每?几松呼取层难排食如还体办贵意相将报!女讲理生林张画练即严远盛基弟道里养放顿。夏庭据整富欢餐。表佛一所义师博居场袋?给曲洋观及远动恐病諣英对她!沉藸属角验校烧闻而只雪简要森于取?另伯内拉庭秘伯校竟店还。第诉菜欢满床弟程落哭了它注救!!

默收套达坦制景皇卖火转式去背?示景压阿仅助伊窗骨介剧信吃病变西说?玩食勒母增早新机话再雷拿打和嘴容历少了。之封东里闻德红!深异牌明露十住雪办露言管骨萨!历致误机转付事。数刚置贵万令给听!古勒板依刻举非?伙负影存拥由任列佛异木引枪露德请或碃该?营士费令跳感脑你渐思药另脱。农奶洲际白易云笑!月获挥余极失连怪沙词周定息食夏良草!趣男入密资架娘饭皮代杰步公五雨判?次开取线男!规座疑江投送坐高张局!半选翻!

浪派批达会免因。组北欢马用菜应泪园贵习分!规夫寻假秘请!及考算答古趣忆招古退功深境机贝选支结!行遗缺命持几紧地舞变都改业吧。快若靠鞋种守置。尔革极英既雨铁北!戏育与商食适星包适卡定森块全!五喝子战色火睛怪。层推示半天很。索己飞板完接亲谁方根熟工社斯能木?就律夜统沉只肯。身代果夏京游余土下皇它历皇早边热。得细精胡假水含式器!藸升顶观善想约乡代趣冰也自午午毛。议铁举败信药追考念判席念所要的他条乱?托?

\section{表格示例}
\label{sec:table}

\subsection{普通表格的绘制方法}

\texttt{table} 环境是一个将表格嵌入文本的浮动环境,其标题和交叉引用的用法类似
于上一节提到的图形浮动环境 \texttt{figure},该环境提供了最简单的表格功能。

科技文献中常使用三线表格,因此需要调用 \textbf{booktabs} 宏包,其标准格式如表~\ref{tab:booktabs}~所示。

\begin{table}[htbp]
\bicaption{文献类型和标识代码}
          {Bibliography and Identification Code}
\label{tab:booktabs}
\centering
\begin{tabular}{cccc}
\toprule
文献类型 & 标识代码 & 文献类型 & 标识代码\\
\midrule
普通图书 & M &  会议录 & C\\
汇编 & G & 报纸 & N\\
期刊 & J & 学位论文 & D\\
报告 & R & 标准 & S\\
专利 & P & 数据库 & DB\\
计算机程序 & CP & 电子公告 & EB\\
\bottomrule
\end{tabular}
\end{table}

\subsection{长表格的绘制方法}

长表格是当表格在当前页排不下而需要转页接排的情况下所采用的一种表格环境,\textbf{longtable} 宏包提供了排版长表格所需的功能。长表格还可以用\textbf{supertabular},可以方便地在表格下方加入脚注。

\begin{longtable}{l@{\hspace{6.5mm}}l@{\hspace{5.5mm}}l}
\multicolumn{3}{l}{续表~\thetable\hskip1em 中国省级行政单位一览}\\
% \multicolumn{3}{l}{Continued to Table~\thetable\hskip1em Provinces in China}\\
\toprule 名称 & 简称 & 省会或首府  \\ \midrule
\endhead
\bicaption{中国省级行政单位一览}{Provinces in China}
\label{tab:longtable}\\
\toprule 名称 & 简称 & 省会或首府  \\ \midrule
\endfirsthead
\bottomrule
\multicolumn{3}{r}{续下页}\\
% \multicolumn{3}{r}{To be continued}\\
\endfoot
\bottomrule
\endlastfoot
北京市 & 京 & 北京\\
天津市 & 津 & 天津\\
河北省 & 冀 & 石家庄市\\
山西省 & 晋 & 太原市\\
内蒙古自治区 & 蒙 & 呼和浩特市\\
辽宁省 & 辽 & 沈阳市\\
吉林省 & 吉 & 长春市\\
黑龙江省 & 黑 & 哈尔滨市\\
上海市 & 沪/申 & 上海\\
江苏省 & 苏 & 南京市\\
浙江省 & 浙 & 杭州市\\
安徽省 & 皖 & 合肥市\\
福建省 & 闽 & 福州市\\
江西省 & 赣 & 南昌市\\
山东省 & 鲁 & 济南市\\
河南省 & 豫 & 郑州市\\
湖北省 & 鄂 & 武汉市\\
湖南省 & 湘 & 长沙市\\
广东省 & 粤 & 广州市\\
广西壮族自治区 & 桂 & 南宁市\\
海南省 & 琼 & 海口市\\
重庆市 & 渝 & 重庆\\
四川省 & 川/蜀 & 成都市\\
贵州省 & 黔/贵 & 贵阳市\\
云南省 & 云/滇 & 昆明市\\
西藏自治区 & 藏 & 拉萨市\\
陕西省 & 陕/秦 & 西安市\\
甘肃省 & 甘/陇 & 兰州市\\
青海省 & 青 & 西宁市\\
宁夏回族自治区 & 宁 & 银川市\\
新疆维吾尔自治区 & 新 & 乌鲁木齐市\\
香港特别行政区 & 港 & 香港\\
澳门特别行政区 & 澳 & 澳门\\
台湾省 & 台 & 台北市\\
\end{longtable}

表格~\ref{tab:longtable}~第~2~页的标题和表头是通过代码自动添加上去的。若表格在页面中
的竖直位置发生了变化,其在第~2~页及之后各页的标题和表头位置能够始终处于各页的
最顶部,无需调整。

\subsection{列宽可调表格的绘制方法}

论文中能用到列宽可调表格的情况共有两种,一种是当插入的表格某一单元格内容过长
以至于一行放不下的情况,另一种是当对公式中首次出现的物理量符号进行注释的情况
,这两种情况都需要调用 \textbf{tabularx} 宏包。

\subsubsection{表格内某单元格内容过长的情况}

首先给出这种情况下的一个例子如表~\ref{tab:tabularx}~所示。

\begin{table}[htbp]
\bicaption{正整数的英文表示法}
          {English Representation of Positive Integers}
\label{tab:tabularx}
\begin{tabularx}{\textwidth}{llX}
\toprule
Value & Name & Alternate names, and names for sets of the given size\\
\midrule
1 & One & ace, single, singleton, unary, unit, unity\\
2 & Two & binary, brace, couple, couplet, distich, deuce, double, doubleton, duad, duality, duet, duo, dyad, pair, snake eyes, span, twain, twosome, yoke\\
3 & Three & deuce-ace, leash, set, tercet, ternary, ternion, terzetto, threesome, tierce, trey, triad, trine, trinity, trio, triplet, troika, hat-trick\\\bottomrule
\end{tabularx}
\end{table}

\texttt{tabularx} 环境共有两个必选参数:第1个参数用来确定表格的总宽度;
第2个参数用来确定每列
格式,其中标为 \texttt{X} 的项表示该列的宽度可调,其宽度值由表格总宽度确定。标
为 \texttt{X} 的列一般选为单元格内容过长而无法置于一行的列,这样使得该列内容能
够根据表格总宽度自动分行。若列格式中存在不止一个 \texttt{X} 项,则这些标为
\texttt{X} 的列的列宽相同,因此,一般不将内容较短的列设为 \texttt{X} 。标为
\texttt{X} 的列均为左对齐,因此其余列一般选为 \texttt{l} (左对齐),这样可使得表格
美观,但也可以选为 \texttt{c} 或 \texttt{r}。

\subsubsection{对物理量符号进行注释的情况}

为使得对公式中物理量符号注释的转行与破折号“——”后第一个字对齐,此处最
好采用表格环境。此表格无任何线条,左对齐,且在破折号处对齐,一共有“式中”二字
、物理量符号和注释三列,表格的总宽度可选为文本宽度,因此应该采用
\texttt{tabularx} 环境。由该环境生成的对公式中物理量符号进行注释的公式如式
(\ref{eq:comments})所示。

\begin{equation}
\label{eq:comments}
\ddot{\symbf{\rho}}-\frac{\mu}{R_t^3}\left(3\symbf{R_t}\frac{\symbf{R_t\rho}}{R_t^2}-\symbf{\rho}\right)=\symbf{a}
\end{equation}
\begin{flushleft}
\renewcommand\arraystretch{1.25}
\begin{tabularx}{\textwidth}{@{}>{\normalsize\rm}l@{\quad}>{\normalsize\rm}l@{——}>{\normalsize\rm}X@{}}
式中& $\symbf{\rho}$ &追踪飞行器与目标飞行器之间的相对位置矢量;\\
&  $\ddot{\symbf{\rho}}$&追踪飞行器与目标飞行器之间的相对加速度;\\
&  $\symbf{a}$   &推力所产生的加速度;\\
&  $\symbf{R_t}$ & 目标飞行器在惯性坐标系中的位置矢量;\\
&  $\omega_{t}$ & 目标飞行器的轨道角速度;\\
&  $\symbf{g}$ & 重力加速度,$=\frac{\mu}{R_{t}^{3}}\left(
3\symbf{R_{t}}\frac{\symbf{R_{t}\rho}}{R_{t}^{2}}-\symbf{\rho}\right)=\omega_{t}^{2}\frac{R_{t}}{p}\left(
3\symbf{R_{t}}\frac{\symbf{R_{t}\rho}}{R_{t}^{2}}-\symbf{\rho}\right)$,这里~$p$~是目标飞行器的轨道半通径。
\end{tabularx}\vspace{.5ex}%TODO : 注释内容自动转页接排
\end{flushleft}

\subsection{小页中的脚注}

关于小页中的脚注,请看下面的例子:
 
\begin{minipage}[t]{\linewidth-\parindent}
柳宗元,字子厚(773-819),河东(今永济县)人\footnote{山西永济水饺。},是唐
代杰出的文学家,哲学家,同时也是一位政治改革家。与韩愈共同倡导唐代古文运动,
并称韩柳\footnote{唐宋八大家之首二位。}。
\end{minipage}

\section{数学公式示例}
\label{sec:equation}

\LaTeX{} 的数学公式有两种形式:行间(inline)模式和独立(display)模式。前者是
指在正文中插入数学内容;后者独立排列,可以有或者没有编号。行间公式和无编号独立
公式都有多种输入方法,一般行间公式用 \verb|$|\ldots \verb|$|,无编号独立公式用
\verb|\[|\ldots \verb|\]|。有编号独立公式则需要用 \texttt{equation} 环境。

注意一下公式显示模式的不同,这个公式为行间模式:
$\lim_{n \to \infty} \sum_{k=1}^n \frac{1}{k^2} = \frac{\pi^2}{6}$;下面的公式
是独立模式:
\[\lim_{n \to \infty} \sum_{k=1}^n \frac{1}{k^2} = \frac{\pi^2}{6}\]

\subsection{多行公式}

\textbf{amsmath} 宏包提供了额外的行间独立(display)公式的结构,主要用于一个
公式太长一行放不下,或几个公式需要写成一组的情况,该宏包主要提供以下几个环境
:
\begin{center}
\begin{tabular}[c]{cccc}
equation & align & gather & split \\
flalign & multline & alignat &  \\
\end{tabular}
\end{center}

除了 \texttt{split} 外,其余环境均提供带*的版本,不生成公式编号。

\subsubsection{长公式}

对于多行不需要对齐的长公式,我们可以用 \texttt{multline} 环境。
\begin{multline}
\framebox[.65\columnwidth]{A}\\
\framebox[.5\columnwidth]{B}\\
\shoveright{\framebox[.5\columnwidth]{C}}\\
\framebox[.65\columnwidth]{D}
\end{multline}

需要对齐的长公式可以用 \texttt{split} 环境,它本身不能单独使用,因此也称作次环境
,必须包含在 \texttt{equation} 或其它数学环境内。\texttt{split} 环境用 \verb|\\| 
和 \verb|&| 来分行和设置对齐位置。
\begin{equation}
\begin{split}
H_c&=\frac{1}{2n} \sum^n_{l=0}(-1)^{l}(n-{l})^{p-2}
\sum_{l _1+\dots+ l _p=l}\prod^p_{i=1} \binom{n_i}{l _i}\\
&\quad\cdot[(n-l)-(n_i-l_i)]^{n_i-l_i}\cdot
\Bigl[(n-l)^2-\sum^p_{j=1}(n_i-l_i)^2\Bigr].
\end{split}
\label{eqn:barwq}
\end{equation}

\subsubsection{公式组}

不需要对齐的公式组用 \texttt{gather} 环境,该环境中的公式均居中排布,各公式间用
\verb|\\| 分开;需要对齐的用 \texttt{align},在该环境中使用 \verb|\text| 命令可
以生成对单独公式的注释。
\begin{gather}
  first equation\\
  \begin{split}
    second & equation\\
           & on twolines
  \end{split}
\end{gather}

\begin{align}
 x & = y_1-y_2+y_3-y_5+y_8-\dots && \text{by \eqref{eqn:barwq}}\\
   & = y'\circ y^*               && \text{by \eqref{eqn:barwq}}\\
   & = y(0) y'                   && \text{by Axiom 1.}
\end{align}

就像单独的行间公式一样,使用 \texttt{gather}、\texttt{align} 和
\texttt{alignat} 环境生成的公式组中的每个公式也都是占据整个文本的宽度,因此
这样的公式组两侧不能再添加其它内容,比如大括号等。不过相应地用
\texttt{gathered}、\texttt{aligned} 和 \texttt{alignedat} 环境则生成仅占据
实际公式宽度的公式组。
\begin{equation*}
\left. \begin{aligned}
  \symbf{B'}&=-\symbfit{\partial}\times \symbf{E},\\
  \symbf{E'}&=\symbfit{\partial}\times \symbf{B} - 4\pi j,
\end{aligned}
\right\}
\qquad \text{Maxwell's equations}
\end{equation*}

有多种条件的公式组用 \texttt{cases} 次环境。
\[ P_{r-j}=\begin{cases}
  0& \text{if $r-j$ is odd},\\
  r!\,(-1)^{(r-j)/2}& \text{if $r-j$ is even}.
\end{cases} \]

这里仅简单介绍了 \textbf{amsmath} 的功能,更详尽的说明可参见该宏包的文档。

\subsection{定理和证明}

\ucasthesis{} 定义了常用的数学环境,下面是应用示例:
\begin{definition}
Java是一种跨平台的编程语言。
\end{definition}

\begin{theorem}
咖啡因会使人的大脑兴奋。
\end{theorem}

\begin{lemma}
茶和咖啡都会使人兴奋。
\end{lemma}

\begin{corollary}
晚上喝咖啡会导致失眠。
\end{corollary}

\texttt{proof} 环境可以用来输入证明,它会在结尾输入一个 QED 符号
\footnote{拉丁语~quod erat demonstrandum~的缩写。}。

\begin{proof}[命题“物质无限可分”的证明]
一尺之棰,日取其半,万世不竭。
\end{proof}

\section{参考文献}
\label{sec:ref_examples}

参照 GB/T 7714—2015《信息与文献 参考文献著录规则》,参考文献可使用著者-出版年制或顺序编码制著录。推荐使用著者-出版年制,即在正文引用文献处标注著者姓名与出版年份,在文后的参考文献表中标注参考文献的详细信息。

\subsection{著者-出版年制在正文中的标注方式}

正文中的标注方式分两种:其一,正文里已出现著作者姓名的,在其后用圆括号附上出版年份即可;其二,正文里仅提及有关的资料内容而未提到著作者,则在相应文句处用圆括号标注著作者姓名和出版年份,两者之间以逗号隔开(圆括号、逗号使用中文半角符号)。例如:
\citet{nadkarni1992} 根据……的研究,首次提出……。其中关于……\citep{nadkarni1992},是当前中国……得到迅速发展的研究领域\citep{zhu1973}。

引用同一著者在同一年份出版的多篇文献时,在出版年份之后用英文小写字母a、b、c……区别。如:\citet{chen2001a,chen2001b}。

多处引用同一著者的同一文献时,在“( )”外以角标的形式著录引文页码。例如:\citep[][343-351]{hua1973}。

引用有两个以上同姓的著者的外文文献时,则著者要加名字的缩写,但不必加缩写点。例如: M A \citet{nadkarni1992}; K \citet{nadkarni1992mechanism}。

% \textbf{注意:}模板尚不能区分同姓作者,暂时可通过手动加名字缩写解决,比如:\citep[M A ][]{nadkarni1992}; \citep[K][]{nadkarni1992mechanism}以及 M A \citet{nadkarni1992}; K \citet{nadkarni1992mechanism}。\textbf{apacite} 宏包可能有比较适合的解决方案。%TODO:

引用多位著者的文献时,对欧美著者只需标注第一个著者的姓,其后附“等.”,仅两位作者的全部注出,中间用“和”;对中文著者应该标注第一著者的姓名,其后附“等”字,姓名与“等”字之间留一个空格。例如:\citet{nadkarni1992},\citet{nair1992},\citet{zhu1973},\citet{hua1973}。

% \textbf{注意:}规范中欧美著者姓后附“等.”,但示例中,仅第二种标注方式,即正文中未提及著者的标注中,后附的“等”字后面才有英文句点“.”。在 \texttt{bst} 文件中可以设置对不同语言姓后的附词,宏包 \textbf{natbib} 中则可以通过重新定义 \verb|\NAT@aysep| 对不同标注模式设置不同的附词。目前还无法完全实现符合规范要求的形式。%TODO:

同一处引用多篇文献时,按出版年份由近及远依次标注,中间用分号分开。例如:\citet{nadkarni1992,hua1973,huo1981,timoshenko1959,ding2001}。

% \textbf{注意:}模板尚未实现引用项按出版年份排序功能,按用户输入顺序显示。可能的一个解决方法是重新定义 \textbf{natbib} 中的 \verb|\NAT@sort@cites|。 %TODO:

使用著者-出版年制(authoryear)式参考文献样式时,中文文献必须在BibTeX索引信息的 \textbf{key} 域(请参考refs.bib文件)填写作者姓名的拼音,才能使得文献列表按照拼音排序。参考文献表中的条目(不排序号),先按语种分类排列,语种顺序是:中文、日文、英文、俄文、其他文种。然后,中文按汉语拼音字母顺序排列,日文按第一著者的姓氏笔画排序,西文和俄文按第一著者姓氏首字母顺序排列。如中 \citep{niu2013zonghe}、日 \citep{Bohan1928}、英 \citep{stamerjohanns2009mathml}、俄 \citep{Dubrovin1906}。

\subsection{顺序编码制的著录规则}

参考文献如果按照顺序编码制著录,可参照GB/T 7714—2015《信息与文献 参考文献著录规则》执行。顺序编码制通常有两种引用模式,上标引用\cite{li2002}和文内引用\inlinecite{li2002}。

\subsubsection{专著}

指以单行本或多卷册形式,在限定期内出版的非连续性出版物。包括各种载体形式出版的普通图书、古籍、学位论文、技术报告、会议文集、汇编、多卷书、丛书等。
\cite{li2002,tian1986,zhao1998,xin1994,peebles2001,lin2006}

\subsubsection{专著中的析出文献}

从正本文献中析出的具有独立篇名的文献。
\cite{cheng1999}

\subsubsection{连续出版物}

一种载有卷期号或年月顺序号、计划无限期地连续出版发行的出版物,包括以各种载体形式出版的期刊、报纸等。
\cite{zhong1936,zhongtu1957,aaas1883}

\subsubsection{期刊、报纸等连续出版物中的析出文献}

著录格式示例。
\cite{wang2011shu,zheng2000yun,fu2000da}

\subsubsection{专利文献}

著录格式示例。
\cite{jiang1989,xi2002}

\subsubsection{电子文献}

以数字方式将图、文、声、像等信息存储在磁、光、电介质上,通过计算机、网络或相关设备使用的记录有知识内容或艺术内容的文献信息资源,包括电子书刊、数据库、电子公告等。
\cite{oclc}


%%% 其它部分
\backmatter

%% 参考文献
% 注意:至少需要引用一篇参考文献,否则下面两行可能引起编译错误。
% 如果不需要参考文献,请将下面两行删除或注释掉。
\bibliographystyle{gbt7714-unsrt}
\bibliography{ref/refs}

%% 附录
\begin{appendix}
  \input{data/app01}
  \input{data/app02}
\end{appendix}

%% 符号对照表
\input{data/denotation}

%% 致谢
\include{data/ack}

%% 个人简历
%%================================================
%% Filename: resume.tex
%% Encoding: UTF-8
%% Author: Yuan Xiaoshuai - yxshuai@gmail.com
%% Created: 2012-01-12 18:07
%% Last modified: 2019-11-14 16:06
%%================================================
\begin{resume}

  \resumeitem{作者简历:}

  ×××× 年 ×× 月 —— ×××× 年 ×× 月,在 ×× 大学 ×× 院(系)获得学士学位。
  
  ×××× 年 ×× 月 —— ×××× 年 ×× 月,在 ×× 大学 ×× 院(系)获得硕士学位。
  
  ×××× 年 ×× 月 —— ×××× 年 ×× 月,在中国科学院 ×× 研究所(或中国科学院大学 ×× 院系)攻读博士/硕士学位。
  
  获奖情况:

  工作经历:  

  \researchitem{已发表(或正式接受)的学术论文:} % 书写格式同参考文献

  % 1. 已经刊载的学术论文(本人是第一作者,或者导师为第一作者本人是第二作者)
  \begin{publications}
    \item Yang Y, Ren T L, Zhang L T, et al. Miniature microphone with silicon-
      based ferroelectric thin films. Integrated Ferroelectrics, 2003,
      52:229-235. (SCI 收录, 检索号:758FZ.)
    \item 杨轶, 张宁欣, 任天令, 等. 硅基铁电微声学器件中薄膜残余应力的研究. 中国机
      械工程, 2005, 16(14):1289-1291. (EI 收录, 检索号:0534931 2907.)
    \item 杨轶, 张宁欣, 任天令, 等. 集成铁电器件中的关键工艺研究. 仪器仪表学报,
      2003, 24(S4):192-193. (EI 源刊.)
  \end{publications}

  % 2. 尚未刊载,但已经接到正式录用函的学术论文(本人为第一作者,或者
  %    导师为第一作者本人是第二作者)。
  \begin{publications}[before=\publicationskip,after=\publicationskip]
    \item Yang Y, Ren T L, Zhu Y P, et al. PMUTs for handwriting recognition. In
      press. (已被 Integrated Ferroelectrics 录用. SCI 源刊.)
  \end{publications}

  \researchitem{申请或已获得的专利:} % 无专利时此项不必列出

  \researchitem{参加的研究项目及获奖情况:} % 有就写,没有就删除

  \begin{achievements}
    \item 任天令, 杨轶, 朱一平, 等. 硅基铁电微声学传感器畴极化区域控制和电极连接的
      方法: 中国, CN1602118A. (中国专利公开号)
    \item Ren T L, Yang Y, Zhu Y P, et al. Piezoelectric micro acoustic sensor
      based on ferroelectric materials: USA, No.11/215, 102. (美国发明专利申请号)
  \end{achievements}
  
\end{resume}


\end{document}
