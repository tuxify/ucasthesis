%%================================================
%% Filename: resume.tex
%% Encoding: UTF-8
%% Author: Yuan Xiaoshuai - yxshuai@gmail.com
%% Created: 2012-01-12 18:07
%% Last modified: 2019-11-14 16:06
%%================================================
\begin{resume}

  \resumeitem{作者简历:}

  ×××× 年 ×× 月 —— ×××× 年 ×× 月,在 ×× 大学 ×× 院(系)获得学士学位。
  
  ×××× 年 ×× 月 —— ×××× 年 ×× 月,在 ×× 大学 ×× 院(系)获得硕士学位。
  
  ×××× 年 ×× 月 —— ×××× 年 ×× 月,在中国科学院 ×× 研究所(或中国科学院大学 ×× 院系)攻读博士/硕士学位。
  
  获奖情况:

  工作经历:  

  \researchitem{已发表(或正式接受)的学术论文:} % 书写格式同参考文献

  % 1. 已经刊载的学术论文(本人是第一作者,或者导师为第一作者本人是第二作者)
  \begin{publications}
    \item Yang Y, Ren T L, Zhang L T, et al. Miniature microphone with silicon-
      based ferroelectric thin films. Integrated Ferroelectrics, 2003,
      52:229-235. (SCI 收录, 检索号:758FZ.)
    \item 杨轶, 张宁欣, 任天令, 等. 硅基铁电微声学器件中薄膜残余应力的研究. 中国机
      械工程, 2005, 16(14):1289-1291. (EI 收录, 检索号:0534931 2907.)
    \item 杨轶, 张宁欣, 任天令, 等. 集成铁电器件中的关键工艺研究. 仪器仪表学报,
      2003, 24(S4):192-193. (EI 源刊.)
  \end{publications}

  % 2. 尚未刊载,但已经接到正式录用函的学术论文(本人为第一作者,或者
  %    导师为第一作者本人是第二作者)。
  \begin{publications}[before=\publicationskip,after=\publicationskip]
    \item Yang Y, Ren T L, Zhu Y P, et al. PMUTs for handwriting recognition. In
      press. (已被 Integrated Ferroelectrics 录用. SCI 源刊.)
  \end{publications}

  \researchitem{申请或已获得的专利:} % 无专利时此项不必列出

  \researchitem{参加的研究项目及获奖情况:} % 有就写,没有就删除

  \begin{achievements}
    \item 任天令, 杨轶, 朱一平, 等. 硅基铁电微声学传感器畴极化区域控制和电极连接的
      方法: 中国, CN1602118A. (中国专利公开号)
    \item Ren T L, Yang Y, Zhu Y P, et al. Piezoelectric micro acoustic sensor
      based on ferroelectric materials: USA, No.11/215, 102. (美国发明专利申请号)
  \end{achievements}
  
\end{resume}
