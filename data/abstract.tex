%%================================================
%% Filename: abstract.tex
%% Encoding: UTF-8
%% Author: Yuan Xiaoshuai - yxshuai@gmail.com
%% Created: 2012-04-24 00:21
%% Last modified: 2019-11-10 10:26
%%================================================
\begin{cabstract}

% 作者所在研究所要求中文摘要内容不能超过一页,稍微多于一页时可以调整行距实现
% \xiaosi[1.72] % fit the abstract in one page

\ucasthesis{}(中国科学院大学学位论文模板)提供中国科学院大学研究生学位论文的模板,代码来源于清华大学学位论文模板(\textsc{ThuThesis}),并根据中国科学院大学学位论文写作规范格式的要求进行了修改。该模板为作者撰写博士论文所用,且通过了格式审查,但由于国科大各研究院所对格式的要求可能存在差异,可能需要根据具体研究院所的格式要求进行一定的调整。

本文的创新点主要有:

  \begin{itemize}
    \item 用例子来解释模板的使用方法;
    \item 用废话来填充无关紧要的部分;
    \item 一边学习摸索一边编写新代码。
  \end{itemize}

\textsf{模板作者注}:关键词分隔符用半角逗号,模板会自动处理替换为《规范》中
规定的分隔符,英文关键词同理。

\end{cabstract}

\ckeywords{TeX/LaTeX, XeLaTeX与中文处理, 科技排版, 国科大, 学位论文
模板, 关于摘要}

\begin{eabstract} 

An abstract of a dissertation is a summary and extraction of research work and
contributions. Included in an abstract should be description of research topic
and research objective, brief introduction to methodology and research
process, and summarization of conclusion and contributions of the research. An
abstract should be characterized by independence and clarity and carry
identical information with the dissertation. It should be such that the
general idea and major contributions of the dissertation are conveyed without
reading the dissertation. 

\end{eabstract}

\ekeywords{TeX/LaTeX, XeLaTeX Chinese, Scientific typesetting system,
Academic thesis template, UCAS, About keywords}
